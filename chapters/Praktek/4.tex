\section{Luthfi Muhammad Nabil/1174035}
\subsection{Soal 1}
Buatlah fungsi pada file chap4\_1174035\_csv.py untuk membuka file csv dengan lib csv mode list : 
\lstinputlisting[firstline=2, lastline=6]{src/4/1174035/Praktek/chap4_1174035_csv.py}

\subsection{Soal 2}
Buatlah fungsi pada file chap4\_1174035\_csv.py untuk membuka file csv dengan lib csv mode dictionary : 
\lstinputlisting[firstline=8, lastline=12]{src/4/1174035/Praktek/chap4_1174035_csv.py}

\subsection{Soal 3}
Buatlah fungsi pada file chap4\_1174035\_pandas.py untuk membuka file csv dengan lib pandas mode list : 
\lstinputlisting[firstline=2, lastline=5]{src/4/1174035/Praktek/chap4_1174035_pandas.py}

\subsection{Soal 4}
Buatlah fungsi pada file chap4\_1174035\_pandas.py untuk membuka file csv dengan lib pandas mode dictionary : 
\lstinputlisting[firstline=6, lastline=10]{src/4/1174035/Praktek/chap4_1174035_pandas.py}

\subsection{Soal 5}
Buatlah fungsi baru di chap4\_1174035\_pandas.py untuk mengubah format tanggal menjadi standard dataframe : 
\lstinputlisting[firstline=11, lastline=13]{src/4/1174035/Praktek/chap4_1174035_pandas.py}

\subsection{Soal 6}
Buatlah fungsi baru di chap4\_1174035\_pandas.py untuk mengubah index kolom : 
\lstinputlisting[firstline=14, lastline=16]{src/4/1174035/Praktek/chap4_1174035_pandas.py}

\subsection{Soal 7}
Buatlah fungsi baru di chap4\_1174035\_pandas.py untuk mengubah atribut atau nama kolom : 
\lstinputlisting[firstline=17, lastline=19]{src/4/1174035/Praktek/chap4_1174035_pandas.py}

\subsection{Soal 8}
Buatlah program chap4\_1174035\_main.py yang menggunakan library chap4\_1174035\_csv.py yang membuat dan membaca file CSV : 
\lstinputlisting{src/4/1174035/Praktek/chap4_1174035_main.py}

\subsection{Soal 9}
Buatlah program chap4\_1174035\_main2.py yang menggunakan library chap4\_1174035\_csv.py yang membuat dan membaca file CSV : 
\lstinputlisting{src/4/1174035/Praktek/chap4_1174035_main2.py}

\subsection{Penanganan Error}
Error yang didapat : KeyError
Deskripsi : Error saat kunci ada yang salah atau tidak ada di dalam file CSV 
\begin{figure}[!htbp]
	\centering
	\includegraphics[height=4cm, width=10cm]{figures/4/1174035/Praktek/chap4_1174035_error.png}
	\caption{Contoh KeyError}
	\label{1174035_Error}
\end{figure}
Penanganan : Menggunakan KeyError seperti pada line berikut : \lstinputlisting{src/4/1174035/Praktek/chap4_1174035_error.py}
\begin{figure}[!htbp]
	\centering
	\includegraphics[height=4cm, width=10cm]{figures/4/1174035/Praktek/chap4_1174035_errorfix.png}
	\caption{Hasil Penanganan Error}
	\label{1174035_ErrorFix}
\end{figure}

\section{Irvan Rizkiansyah/1174043}
	\subsection{Soal 1}
		\lstinputlisting[firstline=10, lastline=14]{src/4/1174043/Praktek/chap4_1174043_csv.py}
	\subsection{Soal 2}
		\lstinputlisting[firstline=16, lastline=20]{src/4/1174043/Praktek/chap4_1174043_csv.py}
	\subsection{Soal 3}
		\lstinputlisting[firstline=10, lastline=12]{src/4/1174043/Praktek/chap4_1174043_pandas.py}
	\subsection{Soal 4}
		\lstinputlisting[firstline=14, lastline=17]{src/4/1174043/Praktek/chap4_1174043_pandas.py}
	\subsection{Soal 5}
		\lstinputlisting[firstline=19, lastline=21]{src/4/1174043/Praktek/chap4_1174043_pandas.py}
	\subsection{Soal 6}
		\lstinputlisting[firstline=23, lastline=26]{src/4/1174043/Praktek/chap4_1174043_pandas.py}
	\subsection{Soal 7}
		\lstinputlisting[firstline=28, lastline=31]{src/4/1174043/Praktek/chap4_1174043_pandas.py}
	\subsection{Soal 8}
		\lstinputlisting[firstline=8, lastline=12]{src/4/1174043/Praktek/chap4_1174043_main.py}
	\subsection{Soal 9}
		\lstinputlisting[firstline=8, lastline=14]{src/4/1174043/Praktek/chap4_1174043_main2.py}
	\subsection{Penanganan Error}
		\lstinputlisting[firstline=8, lastline=17]{src/4/1174043/Praktek/chap4_1174043_error.py}
		\begin{figure} [ht]
			\centerline{\includegraphics[width=0.6\textwidth]{figures/4/1174043/Praktek/fix_error.png}}
			\caption{Fix Error}
			\label{Fix Error}
		\end{figure}

		\ref{fix_error}