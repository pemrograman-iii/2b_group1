\section{Hagan Rowlenstino/1174040}
	\subsection{Soal 1}
	Bar :

	\lstinputlisting{src/6/1174040/Praktek/chap6_1174040_bar.py}

	\subsection{Soal 2}

	Scatter :

	\lstinputlisting{src/6/1174040/Praktek/chap6_1174040_scatter.py}

	\subsection{Soal 3}

	Pie :

	\lstinputlisting{src/6/1174040/Praktek/chap6_1174040_pie.py}

	\subsection{Soal 4}

	Plot :

	\lstinputlisting{src/6/1174040/Praktek/chap6_1174040_plot.py}

	\subsection{Penanganan Error}
	Error yang ditemukan yaitu ValueError. cara penanggulangan nya yatu dengan mengecek kembali panjang dan lebar subplotnya agar tidak kurang dari urutan yang kita buat.

	\lstinputlisting{src/6/1174040/Praktek/chap6_1174040_err.py}

\section{Faisal Najib Abdullah 1174042}
\subsection{Praktek}
\subsubsection{Soal No. 1}
\hfill \break
Buatlah librari fungsi (file terpisah/library dengan nama NPMbar.py) untuk plot dengan jumlah subplot adalah NPM mod 3 + 2!

\hfill \break
\textbf{Kode Program}

\lstinputlisting[caption = Kode program membuat fungsi Bar Plot menggunakan Matplotlib., firstline=1, lastline=21]{src/6/1174042/1174042_bar.py}

\hfill \break
\textbf{Hasil Compile}

\begin{figure}[H]
	\includegraphics[width=12cm]{figures/6/1174042/p1.png}
	\centering
	\caption{Hasil compile membuat fungsi Bar Plot menggunakan Matplotlib.}
\end{figure}

\subsubsection{Soal No. 2}
\hfill \break
Buatlah librari fungsi (file terpisah/library dengan nama NPMscatter.py) untuk plot dengan jumlah subplot NPM mod 3 + 2!

\hfill \break
\textbf{Kode Program}

\lstinputlisting[caption = Kode program membuat fungsi Scatter Plot menggunakan Matplotlib., firstline=1, lastline=23]{src/6/1174042/1174042_scatter.py}

\hfill \break
\textbf{Hasil Compile}

\begin{figure}[H]
	\includegraphics[width=12cm]{figures/6/1174042/p2.png}
	\centering
	\caption{Hasil compile membuat fungsi Scatter Plot menggunakan Matplotlib.}
\end{figure}

\subsubsection{Soal No. 3}
\hfill \break
Buatlah librari fungsi (file terpisah/library dengan nama NPMpie.py) untuk plot dengan jumlah subplot NPM mod 3 + 2!

\hfill \break
\textbf{Kode Program}

\lstinputlisting[caption = Kode program membuat fungsi Pie Plot menggunakan Matplotlib., firstline=1, lastline=23]{src/6/1174042/1174042_pie.py}

\hfill \break
\textbf{Hasil Compile}

\begin{figure}[H]
	\includegraphics[width=12cm]{figures/6/1174042/p3.png}
	\centering
	\caption{Hasil compile membuat fungsi Pie Plot menggunakan Matplotlib.}
\end{figure}

\subsubsection{Soal No. 4}
\hfill \break
Buatlah librari fungsi (file terpisah/library dengan nama NPMplot.py) untuk plot dengan jumlah subplot NPM mod 3 + 2

\hfill \break
\textbf{Kode Program}

\lstinputlisting[caption = Kode program membuat fungsi Plot menggunakan Matplotlib., firstline=1, lastline=23]{src/6/1174042/1174042_plot.py}

\hfill \break
\textbf{Hasil Compile}

\begin{figure}[H]
	\includegraphics[width=12cm]{figures/6/1174042/p4.png}
	\centering
	\caption{Hasil compile membuat fungsi Plot menggunakan Matplotlib.}
\end{figure}


\subsection{Penanganan Error}
Tuliskan  peringatan  error  yang  didapat  dari  mengerjakan  praktek  keenam  ini, dan  jelaskan  cara  penanganan  error  tersebut. dan  Buatlah  satu  fungsi  yang menggunakan try except untuk menanggulangi error tersebut.

\hfill \break
Peringatan error di praktek kelima ini, yaitu:
\begin{itemize}
	\item Syntax Errors
	Syntax Errors adalah suatu keadaan saat kode python mengalami kesalahan penulisan. Solusinya adalah memperbaiki penulisan kode yang salah.
	
	\item Name Error
	NameError adalah exception yang terjadi saat kode melakukan eksekusi terhadap local name atau global name yang tidak terdefinisi. Solusinya adalah memastikan variabel atau function yang dipanggil ada atau tidak salah ketik.
	
	\item Type Error
	TypeError adalah exception yang akan terjadi apabila pada saat dilakukannya eksekusi terhadap suatu operasi atau fungsi dengan type object yang tidak sesuai. Solusi dari error ini adalah mengkoversi varibelnya sesuai dengan tipe data yang akan digunakan.
\end{itemize}
\hfill \break
Fungsi yang menggunakan try except untuk menanggulangi error.

\hfill \break
\textbf{Kode Program}

\lstinputlisting[caption = Kode program membuat fungsi penanganan error., firstline=161, lastline=178]{src/6/1174042/1174042.py}

\hfill \break
\textbf{Hasil Compile}

\section{Irvan Rizkiansyah/1174043}
	\subsection{Nomor 1}
	Histogram :
		\lstinputlisting{src/6/1174043/Praktek/chap6_1174043_hist.py}
	\subsection{Nomor 2}
	Scatter :
		\lstinputlisting{src/6/1174043/Praktek/chap6_1174043_scatter.py}
	\subsection{Nomor 3}
	Pie :
		\lstinputlisting{src/6/1174043/Praktek/chap6_1174043_pie.py}
	\subsection{Nomor 4}
	Plot :
		\lstinputlisting{src/6/1174043/Praktek/chap6_1174043_plot.py}
	\subsection{Penanganan Error}
		\lstinputlisting{src/6/1174043/Praktek/chap6_1174043_error.py}

\section{Luthfi Muhammad Nabil/1174035}
	\subsection{Soal 1}
	Buatlah librari fungsi dengan nama file chap6\_1174035\_bar.py denga jumlah subplot adalah NPM mod 3+2 : 

	\lstinputlisting{src/6/1174035/Praktek/chap6_1174035_bar.py}
	
	Hasil dari kompilasi terdapat pada gambar \ref{1174035_Bar}
	\begin{figure}[ht]
		\centerline{\includegraphics[width=0.8\textwidth]{figures/6/1174035/Praktek/Bar.png}}
		\caption{Grafik Bar}
		\label{1174035_Bar}
	\end{figure}
	\subsection{Soal 2}

	Buatlah librari fungsi dengan nama file chap6\_1174035\_scatter.py denga jumlah subplot adalah NPM mod 3+2 : 

	\lstinputlisting{src/6/1174035/Praktek/chap6_1174035_scatter.py}
	
	Hasil dari kompilasi terdapat pada gambar \ref{1174035_Scatter}
	\begin{figure}[ht]
		\centerline{\includegraphics[width=0.8\textwidth]{figures/6/1174035/Praktek/Scatter.png}}
		\caption{Grafik Scatter}
		\label{1174035_Scatter}
	\end{figure}
	
	\subsection{Soal 3}

	Buatlah librari fungsi dengan nama file chap6\_1174035\_pie.py denga jumlah subplot adalah NPM mod 3+2 : 

	\lstinputlisting{src/6/1174035/Praktek/chap6_1174035_pie.py}
	
	Hasil dari kompilasi terdapat pada gambar \ref{1174035_Pie}
	\begin{figure}[ht]
		\centerline{\includegraphics[width=0.8\textwidth]{figures/6/1174035/Praktek/Pie.png}}
		\caption{Grafik Pie}
		\label{1174035_Pie}
	\end{figure}
	\subsection{Soal 4}

	Buatlah librari fungsi dengan nama file chap6\_1174035\_plot.py denga jumlah subplot adalah NPM mod 3+2 : 

	\lstinputlisting{src/6/1174035/Praktek/chap6_1174035_plot.py}
	
	Hasil dari kompilasi terdapat pada gambar \ref{1174035_Plot}
	\begin{figure}[ht]
		\centerline{\includegraphics[width=0.8\textwidth]{figures/6/1174035/Praktek/Plot.png}}
		\caption{Grafik Plot}
		\label{1174035_Plot}
	\end{figure}
	\subsection{Penanganan Error}
	Error yang ditemukan yaitu ValueError. Cara penanggulangan nya yatu dengan mengecek kembali panjang dan lebar subplotnya dan memastikan jika urutan pada subplot (Angka ke 3) tidak kurang atau lebih dari panjang*lebar yang ditentukan.
	Berikut Penerapan dari try except terdapat pada gambar \ref{1174035_Error}
	\lstinputlisting{src/6/1174035/Praktek/chap6_1174035_error.py}
	Hasil dari kompilasi (Jika Error) terdapat pada gambar \ref{1174035_Error} 
	\begin{figure}[ht]
		\centerline{\includegraphics[width=0.8\textwidth]{figures/6/1174035/Praktek/Error.png}}
		\caption{Hasil Kompilasi Error}
		\label{1174035_Error}
	\end{figure}
	
	
	\subsection{Dika Sukma Pradana 1174050}
\subsection{Soal 1}

\lstinputlisting[firstline=1, lastline=24]{src/6/1174050/Praktek/p1174050_bar.py}
untuk memunculkan hasilnya kita dapat memanggil menggunakan :
\lstinputlisting[firstline=1, lastline=1]{src/6/1174050/Praktek/main.py}
\lstinputlisting[firstline=6, lastline=6]{src/6/1174050/Praktek/main.py}
dan akan menghasilkan grafik seperti gambar berikut:
\begin{figure}[H]
\centering
\includegraphics[width=7cm]{figures/6/1174050/Praktek/pbar.png}
\caption{Grafik Batang}
\end{figure}

\subsection{Soal 2}

\lstinputlisting[firstline=1, lastline=22]{src/6/1174050/Praktek/p1174050_scatter.py}
untuk memunculkan hasilnya kita dapat memanggil menggunakan :
\lstinputlisting[firstline=2, lastline=2]{src/6/1174050/Praktek/main.py}
\lstinputlisting[firstline=7, lastline=7]{src/6/1174050/Praktek/main.py}
dan akan menghasilkan grafik seperti gambar berikut:
\begin{figure}[H]
\centering
\includegraphics[width=7cm]{figures/6/1174050/Praktek/pscatter.png}
\caption{Grafik Scatter}
\end{figure}

\subsection{Soal 3}

\lstinputlisting[firstline=1, lastline=46]{src/6/1174050/Praktek/p1174050_pie.py}
untuk memunculkan hasilnya kita dapat memanggil menggunakan :
\lstinputlisting[firstline=3, lastline=3]{src/6/1174050/Praktek/main.py}
\lstinputlisting[firstline=8, lastline=8]{src/6/1174050/Praktek/main.py}
dan akan menghasilkan grafik seperti gambar berikut:
\begin{figure}[H]
\centering
\includegraphics[width=7cm]{figures/6/1174050/Praktek/ppie.png}
\caption{Grafik Pie}
\end{figure}

\subsection{Soal 4}
\lstinputlisting[firstline=1, lastline=22]{src/6/1174050/Praktek/p1174050_plot.py}
untuk memunculkan hasilnya kita dapat memanggil menggunakan :
\lstinputlisting[firstline=4, lastline=4]{src/6/1174050/Praktek/main.py}
\lstinputlisting[firstline=9, lastline=9]{src/6/1174050/Praktek/main.py}
dan akan menghasilkan grafik seperti gambar berikut:
\begin{figure}[H]
\centering
\includegraphics[width=7cm]{figures/6/1174050/Praktek/pplot.png}
\caption{Grafik Plot}
\end{figure}


\subsection{Penanganan Error}
Macam - macam error :
\begin{itemize}
	\item Syntax Errors
	Syntax Errors adalah suatu keadaan saat kode python mengalami kesalahan penulisan. Solusinya adalah memperbaiki penulisan kode yang salah.
	
	\item Name Error
	NameError adalah exception yang terjadi saat kode melakukan eksekusi terhadap local name atau global name yang tidak terdefinisi. Solusinya adalah memastikan variabel atau function yang dipanggil ada atau tidak salah ketik.
	
	\item Type Error
	TypeError adalah exception yang akan terjadi apabila pada saat dilakukannya eksekusi terhadap suatu operasi atau fungsi dengan type object yang tidak sesuai. Solusi dari error ini adalah mengkoversi varibelnya sesuai dengan tipe data yang akan digunakan.
\end{itemize}
Apabila terjadi suatu ke-eror-an maka dapat ditangani dengan cara sebagai berikut :
\lstinputlisting[firstline=1, lastline=7]{src/6/1174050/Praktek/error.py}


\section{Ichsan Hizman Hardy 1174034}
\subsection{Praktek}
\subsubsection{soal 1}
 Buatlah librari fungsi (file terpisah/library dengan nama NPM bar.py) untuk plot dengan jumlah subplot adalah NPM mod 3 + 2
\lstinputlisting[firstline=8, lastline=28]{src/6/1174034/Praktek/p1174034_bar.py}
untuk memunculkan hasilnya kita dapat memanggil menggunakan :
\lstinputlisting[firstline=8, lastline=8]{src/6/1174034/Praktek/main_san.py}
\lstinputlisting[firstline=13, lastline=13]{src/6/1174034/Praktek/main_san.py}
dan akan menghasilkan grafik seperti gambar berikut:
\begin{figure}[H]
\centering
\includegraphics[width=7cm]{figures/6/1174034/Praktek/p1.png}
\caption{Grafik Batang}
\label{Ichsan}
\end{figure}

\subsubsection{soal 2}
Buatlah librari fungsi (file terpisah/library dengan nama NPM scatter.py) untuk plot dengan jumlah subplot NPM mod 3 + 2
\lstinputlisting[firstline=8, lastline=28]{src/6/1174034/Praktek/p1174034_titik.py}
untuk memunculkan hasilnya kita dapat memanggil menggunakan :
\lstinputlisting[firstline=9, lastline=9]{src/6/1174034/Praktek/main_san.py}
\lstinputlisting[firstline=14, lastline=14]{src/6/1174034/Praktek/main_san.py}
dan akan menghasilkan grafik seperti gambar berikut:
\begin{figure}[H]
\centering
\includegraphics[width=7cm]{figures/6/1174034/Praktek/p2.png}
\caption{Grafik Titik}
\label{Ichsan}
\end{figure}


\subsubsection{soal 3}
Buatlah librari fungsi (file terpisah/library dengan nama NPM pie.py) untuk plot dengan jumlah subplot NPM mod 3 + 2
\lstinputlisting[firstline=8, lastline=50]{src/6/1174034/Praktek/p1174034_pie.py}
untuk memunculkan hasilnya kita dapat memanggil menggunakan :
\lstinputlisting[firstline=10, lastline=10]{src/6/1174034/Praktek/main_san.py}
\lstinputlisting[firstline=15, lastline=15]{src/6/1174034/main_san.py}
dan akan menghasilkan grafik seperti gambar berikut:
\begin{figure}[H]
\centering
\includegraphics[width=7cm]{figures/6/1174034/Praktek/p3.png}
\caption{Grafik Pie}
\label{Ichsan}
\end{figure}


\subsubsection{soal 4}
Buatlah librari fungsi (file terpisah/library dengan nama NPM plot.py) untuk plot dengan jumlah subplot NPM mod 3 + 2
\lstinputlisting[firstline=8, lastline=28]{src/6/1174034/Praktek/p1174034_plot.py}
untuk memunculkan hasilnya kita dapat memanggil menggunakan :
\lstinputlisting[firstline=11, lastline=11]{src/6/1174034/Praktek/main_san.py}
\lstinputlisting[firstline=16, lastline=16]{src/6/1174034/Praktek/main_san.py}
dan akan menghasilkan grafik seperti gambar berikut:
\begin{figure}[H]
\centering
\includegraphics[width=7cm]{figures/6/1174034/Praktek/p4.png}
\caption{Grafik Plot}
\label{Ichsan}
\end{figure}

\subsection{Penanganan Eror}
Apabila terjadi suatu ke-eror-an maka dapat ditangani dengan cara sebagai berikut :
\lstinputlisting[firstline=8, lastline=14]{src/6/1174034/Praktek/eror.py}