\section{Hagan Rowlenstino/1174040}
	\subsection{Soal 1} 
		\begin{itemize}
			\item Device Manager : Seperti namanya sendiri, device manager berfungsi untuk menampilkan dan mengelola semua hardware yang terinstall ataupun dapat di instalasi ke dalam windows.

			\item folder /dev : Di dalam sistem operasi Linux, perangkat yang tehubung akan dianggap sebagai file. di dalam folder /dev inilah file - file  tersebut berada.
		\end{itemize}

	\subsection{Soal 2}
	Langkah - langkah instalasi driver arduino :
		\begin{enumerate}
			\item download file driver arduino terlebih dahulu dan masukkan ke dalam directory yang diinginkan
			\item hubungkan arduinio uno anda ke pc anda dengan kabel USB yang tersedia
			\item lalu windows akan memunculkan pop up yang memberitahu bahwa ingin menginstall dirver, tapi nanti tidak akan menemukan drivernya
			\item buka Device Manager 
			\item cari unknown device di dalam Device Manager di dalam tab other device
			\item klik kanan pada unknown device tersebut lalu pilih update driver software
			\item pilih browse my computer for driver software lalu masukkan directory dimana anda menyimpan driver arduino yang telah anda download tadi
			\item setelah itu klik install dan tunggu hingga proses selesai
			\item arduino pun sudah terbaca di pc anda 
		\end{enumerate}

	\subsection{Soal 3}
	Untuk melihat atau membaca baudrate dan port kita hanya perlu menginstall Arduino IDE, setelah itu buka menu serial monitor yang berada di tab tools. Dari sana akan terlihat baik baudrate dan port yang sedang digunakan oleh arduin anda.

	\subsection{Soal 4}
	PySerial merupakan sebuah library yang digunakan untuk komunikasi ke port serial terutama untuk mikrokontroller. PySerial pertama kali diluncurkan pada tahun 2002 yang makin berkembang dalam setiap versinya hingga tahun 2017 lalu.

	\subsection{Soal 5}
		\begin{itemize}
			\item \begin{verbatim}stop()\end{verbatim} : untuk menghentikan pembacaan program
			\item \begin{verbatim}serial.to_bytes(sequence)\end{verbatim} : berfungsi untuk mengubah sequence ke dalam bytes agar dapat dikirim ke dalam arduino.
			\item \begin{verbatim}close()\end{verbatim} : untuk menutup port dan menghentikan pembacaan program
		\end{itemize}

	\subsection{Soal 6}
	Dengan menggunakan pengulangan kita dapat mengambil data berkali - kali tanpa harus mengeksekusi file python tersebut berulang - ulang. Tanpa perulangan juga penting karena dapat digunakan di saat saat tertentu seperti jika ingin mengukur suhu ruangan yang hanya dilakukan pada saat saat tertentu tidak terus menerus.

	\subsection{Soal 7}
	Untuk membuat fungsi yang menggunakan pyserial kita hanya perlu untuk menginisialisasi pembubatan funsi dengan menggunakan def namafungsi() : lalu masukkan pyserial tersebut dengan indentasi. atau cukup dengan menggunakan fungsi while loop degan menggunakan while true:
